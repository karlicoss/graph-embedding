\documentclass[a4paper, 10pt]{article}

\usepackage[english, russian]{babel}

\usepackage[utf8]{inputenc}

%\usepackage{times}
\usepackage{setspace}
\usepackage{indentfirst}

\begin{document}

\title{Гамма-алгоритм}
\author{Дмитрий Герасимов}
\date{2011}

\maketitle
\tableofcontents

\section{Алгоритм}

\begin{enumerate}
    \item На входе имеем граф $G$
    \item Разделим граф на компоненты связности. Каждую из них
          можно уложить независимо от остальных.
    \item Разделим полученные графы на компоненты вершинной
          двусвязности. Так как граф блоков-точек сочленения — дерево,
          можно укладывать их по очереди, укладывая блок внутри соответствующей
          грани смежного ему.
    \item Разделим полученные графы на компоненты рёберной
          двусвязности. Граф компонент рёберной двусвязности — дерево,
          аналогично, будем укладывать их по очереди.
    \item 
          \begin{enumerate}
          \item Граф состоит из одной вершины, тогда его укладка тривиальна.
          \item Граф связный и не может быть деревом по предыдущим шагам.
                Тогда в нём найдётся простой цикл.
          \end{enumerate}
    \item Пусть $F$ — множество граней графа, $outerFace$ — указатель на внешнюю его грань, $S$ — множество его сегментов.
    \item Найдём любой простой цикл $c$. Добавим $c$ в $F$, $outerFace$ = $c$. Пересчитаем $S$.
    \item
          \begin{enumerate}
          \item $S$ — пусто, тогда граф может быть уложен, а все его грани нами уже получены.
          \item $S$ — не пусто.
          \end{enumerate}
    \item Для каждого сегмента $s$ в $S$ посчитаем $K(s)$ — количество граней, совместных с $s$.
    \item Пусть $ms$ — грань такая, что $K(ms)$ — минимально среди всех $K(s)$.
          \begin{enumerate}
          \item $K(ms) = 0$, тогда сегмент $ms$ не может быть уложен, а, следовательно, граф не планарен.
          \item $K(ms) \ne 0$, тогда выберем любую грань $f$, совместную с $ms$. Найдём произвольную простую цепь $a$ в $ms$ между двумя уже уложенными вершинами. Уложим её, разобьём $f$ на две грани и удалим $ms$ из $S$.
          \end{enumerate}
    \item Перейдём к шагу 8.
        
    
\end{enumerate}

\section{Концепция}
выаывпывпывп:w


\end{document}
