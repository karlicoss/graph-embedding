\documentclass[a4paper, 10pt]{article}

\usepackage[english, russian]{babel}

\usepackage[utf8]{inputenc}

%\usepackage{times}
\usepackage{setspace}
\usepackage{indentfirst}
\usepackage{amsthm}
\usepackage{theoremref}
\usepackage{hyperref}

\newtheorem{mydefinition}{Определение}
\newtheorem{mytheorem}{Теорема}
\newtheorem{mylemma}{Лемма}

\begin{document}

\title{Проверка графа на планарность и его укладка(«Гамма-алгоритм»)}
\author{Дмитрий Герасимов}
\date{2011}

\maketitle

\section{Определения}
{\bf Планарным}({\it plane}) называется граф, который может быть изображён на плоскости без пересечений его рёбер друг с другом.

{\bf Плоской укладкой}({\it planar embedding}) называется изображение планарного графа на плоскости.

{\bf Гранью}({\it face}) называется такая область, каждые две точки которой могут быть соединены жордановой кривой, не пересекающей рёбра графа. Очевидно, на плоскости может быть выделена единственная неограниченная {\bf внешняя}({\it outer}) грань.

\section{Поставновка задачи}

\section{Теоремы}

\begin{mylemma}\thlabel{econtree}
Граф компонент реберной двусвязности — дерево.
\begin{proof}
\href{http://neerc.ifmo.ru/mediawiki/index.php/%D0%93%D1%80%D0%B0%D1%84_%D0%BA%D0%BE%D0%BC%D0%BF%D0%BE%D0%BD%D0%B5%D0%BD%D1%82_%D1%80%D0%B5%D0%B1%D0%B5%D1%80%D0%BD%D0%BE%D0%B9_%D0%B4%D0%B2%D1%83%D1%81%D0%B2%D1%8F%D0%B7%D0%BD%D0%BE%D1%81%D1%82%D0%B8}{вики-конспекты}
\end{proof}
\end{mylemma}

\begin{mylemma}\thlabel{vcontree}
Граф блоков-точек сочленения — дерево.
\begin{proof}
\href{http://neerc.ifmo.ru/mediawiki/index.php/%D0%93%D1%80%D0%B0%D1%84_%D0%B1%D0%BB%D0%BE%D0%BA%D0%BE%D0%B2-%D1%82%D0%BE%D1%87%D0%B5%D0%BA_%D1%81%D0%BE%D1%87%D0%BB%D0%B5%D0%BD%D0%B5%D0%BD%D0%B8%D1%8F}{вики-конспекты}
\end{proof}
\end{mylemma}

\begin{mylemma}\thlabel{anyedge}
Если граф планарен, то для любого его ребра существует такая укладка, что это ребро лежит на внешней грани.
\end{mylemma}

\begin{mylemma}
Если компоненты связности графа планарны, то и граф — планарен.
\begin{proof}
Просто уложим графы достаточно далеко друг от друга.
\end{proof}
\end{mylemma}

\begin{mylemma}
Если компоненты реберной двусвязности графа планарны, то и граф планарен.
\begin{proof}
\href{http://neerc.ifmo.ru/mediawiki/index.php/%D0%A3%D0%BA%D0%BB%D0%B0%D0%B4%D0%BA%D0%B0_%D0%B3%D1%80%D0%B0%D1%84%D0%B0_%D1%81_%D0%BF%D0%BB%D0%B0%D0%BD%D0%B0%D1%80%D0%BD%D1%8B%D0%BC%D0%B8_%D0%BA%D0%BE%D0%BC%D0%BF%D0%BE%D0%BD%D0%B5%D0%BD%D1%82%D0%B0%D0%BC%D0%B8_%D1%80%D0%B5%D0%B1%D0%B5%D1%80%D0%BD%D0%BE%D0%B9_%D0%B4%D0%B2%D1%83%D1%81%D0%B2%D1%8F%D0%B7%D0%BD%D0%BE%D1%81%D1%82%D0%B8}{вики-конспекты}
\end{proof}
\end{mylemma}

\begin{mylemma}
Если компоненты вершинной двусвязности графа планарны, то и граф планарен.
\begin{proof}
\href{http://neerc.ifmo.ru/mediawiki/index.php/%D0%A3%D0%BA%D0%BB%D0%B0%D0%B4%D0%BA%D0%B0_%D0%B3%D1%80%D0%B0%D1%84%D0%B0_%D1%81_%D0%BF%D0%BB%D0%B0%D0%BD%D0%B0%D1%80%D0%BD%D1%8B%D0%BC%D0%B8_%D0%BA%D0%BE%D0%BC%D0%BF%D0%BE%D0%BD%D0%B5%D0%BD%D1%82%D0%B0%D0%BC%D0%B8_%D0%B2%D0%B5%D1%80%D1%88%D0%B8%D0%BD%D0%BD%D0%BE%D0%B9_%D0%B4%D0%B2%D1%83%D1%81%D0%B2%D1%8F%D0%B7%D0%BD%D0%BE%D1%81%D1%82%D0%B8}{вики-конспекты}
\end{proof}
\end{mylemma}

\section{Методы решения}
Существует несколько методов тестирования планарности:
\begin{itemize}
    \item Можно использовать критерий Понтрягина-Куратовского и искать подграф, стягиваемый к $K_5$ или $K_{3,3}$, но, очевидно, это очень трудоёмко.
    \item Если у нас получена планарная укладка, очевидно, убрав одну вершину, мы также получим планарную укладку, но меньшего графа. Можно предположить, что можно, наоборот, получать планарную укладку добавлением вершин — на этом основываются алгоритмы, основанные на добавлении вершин, но они сложны и здесь не рассматриваются.
    \item Также выделяют семейство алгоритмов ??({\it cycle-based}) — . Один из них — алгоритм, предложенный Auslander и Parter(1961), Goldstein(1963) и Bader(1964), который и называется гамма-алгоритмом.
\end{itemize}

\section{Гамма-алгоритм}

Введём несколько дополнительных определений:

\begin{mydefinition}
Пусть $L$ — уже уложенный подграф $G$. Тогда {\bf сегментом}({\it segment}) $s$ называется:
\begin{itemize}
\item ребро, оба конца которого принадлежат $L$, но которое $L$ не принадлежит
\item компонента связности графа $G - L$, дополненная рёбрами, ведущими из неё в $L$
\end{itemize}
\end{mydefinition}

\begin{mydefinition}
{\bf Контактными вершинами}({\it attachments}) сегмента $s$ называются его вершины, также принадлежащие к уже уложенной части графа.
\end{mydefinition}

\begin{mydefinition}
Грань $f$ {\bf вмещает} сегмент $s$, если все контактные вершины $s$ принадлежат $f$.
\end{mydefinition}

Алогритм:
\begin{enumerate}
    \item На входе имеем граф $G$
    \item Разделим граф на компоненты связности.
    \item Разделим полученные графы на компоненты вершинной двусвязности.
    \item Разделим полученные графы на компоненты рёберной двусвязности. 
    \item 
          \begin{enumerate}
          \item Граф состоит из одной вершины, тогда его укладка тривиальна.
          \item Граф связный и не может быть деревом по предыдущим шагам.
                Тогда в нём найдётся простой цикл.
          \end{enumerate}
    \item Пусть $F$ — множество граней графа, $outerFace$ — указатель на внешнюю его грань, $S$ — множество его сегментов.
    \item Найдём любой простой цикл $c$. Добавим $c$ в $F$, $outerFace$ = $c$. Пересчитаем $S$.
    \item
          \begin{enumerate}
          \item $S$ — пусто, тогда граф может быть уложен, а все его грани нами уже получены.
          \item $S$ — не пусто.
          \end{enumerate}
    \item Для каждого сегмента $s$ в $S$ посчитаем $K(s)$ — количество граней, совместных с $s$.
    \item Пусть $ms$ — грань такая, что $K(ms)$ — минимально среди всех $K(s)$.
          \begin{enumerate}
          \item $K(ms) = 0$, тогда сегмент $ms$ не может быть уложен, а, следовательно, граф не планарен.
          \item $K(ms) \ne 0$, тогда выберем любую грань $f$, совместную с $ms$. Найдём произвольную простую цепь $a$ в $ms$ между двумя уже уложенными вершинами. Уложим её, разобьём $f$ на две грани и удалим $ms$ из $S$.
          \end{enumerate}
    \item Перейдём к шагу 8.
        
    
\end{enumerate}

\section{Концепция}
выаывпывпывп:w


\end{document}
